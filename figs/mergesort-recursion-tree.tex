% Red-black tree
% Author: Madit
\documentclass{article}
\usepackage{tikz}
%%%<
\usepackage{verbatim}
\usepackage[active,tightpage]{preview}
\usetikzlibrary{calc}
\PreviewEnvironment{tikzpicture}
\setlength{\PreviewBorder}{10pt}%
%%%>
\begin{comment}
:Title: Red-black tree
:Tags: Trees;Graphs
:Author: Madit
:Slug: red-black-tree

A red-black tree is a special type of binary tree, used in computer science
to organize pieces of comparable data, such as text fragments or numbers.
(Wikipedia)
\end{comment}
\usetikzlibrary{arrows}

\tikzset{
  treenode/.style = {align=center, inner sep=0pt, text centered,
    font=\sffamily},
  arn_n/.style = {treenode, circle, white, font=\sffamily\bfseries, draw=black,
    fill=black, text width=1.5em},% arbre rouge noir, noeud noir
  arn_r/.style = {treenode, circle, black, draw=black, 
    text width=1.5em, very thick},% arbre rouge noir, noeud rouge
  arn_x/.style = {treenode, rectangle, text width=1.5 em, %draw=black,
    minimum width=1.5em, minimum height=1.5em}% arbre rouge noir, nil
}

\begin{document}
% \begin{tikzpicture}[<-,>=stealth',level/.style={sibling distance = 5cm/#1,
%   level distance = 1.5cm}] 
% \node [arn_r] {28}
%     child{ node [arn_r] {6} 
%             child{ node [arn_r] {1} 
%             	child{ node [arn_r] {0}}% edge from parent node[above left]
%                          %{$x$}} %for a named pointer
% 							child{ node [arn_r] {1}}
%             }
%             child{ node [arn_r] {5}
% 							child{ node [arn_r] {2}}
% 							child{ node [arn_r] {3}}
%             }                            
%     }
%     child{ node [arn_r] {22}
%             child{ node [arn_r] {9} 
% 							child{ node [arn_r] {4}}
% 							child{ node [arn_r] {5}}
%             }
%             child{ node [arn_r] {13}
% 							child{ node [arn_r] {6}}
% 							child{ node [arn_r] {7}}
%             }
% 		}
% ; 
% \end{tikzpicture}
\begin{tikzpicture}[level/.style={sibling distance = 3cm/#1,
  level distance = 1.5cm}] 
 % \draw (0,0) node {upsweep};
 \node [arn_x]{$\!\!\!\! merge(n)$}
child {
      node [arn_x]{\small\!\! $merge(\frac{n}{2}$)}
        child { 
          node (B) [arn_x]{\small\!\! $merge(\frac{n}{4})$}
            child {
              node (BA) [arn_x]{$\ldots$}
                child {
                  node [arn_x]{$\frac{n}{2^h}$}
                }
              }
            child {
              node (BB) [arn_x]{$\ldots$}
                child {
                  node [arn_x]{$\frac{n}{2^h}$}
                }
              }
        }
        child {
          node (C) [arn_x]{\small\!\! $merge(\frac{n}{4})$}
            child {
              node (CA) [arn_x]{$\ldots$}
              child {
                  node [arn_x]{$\frac{n}{2^h}$}
                }
              }
            child {
              node (CB) [arn_x]{$\ldots$}
              child {
                  node [arn_x]{$\frac{n}{2^h}$}
                }
              }
        }
      }
child {
  node [arn_x]{\small $merge(\frac{n}{2})$}
    child { 
      node (D) [arn_x]{\small\!\! $merge(\frac{n}{4})$}
        child {
          node (DA) [arn_x]{$\ldots$}
          child {
                  node [arn_x]{$\frac{n}{2^h}$}
                }
          }
        child {
          node (DB) [arn_x]{$\ldots$}
          child {
                  node [arn_x]{$\frac{n}{2^h}$}
                }
          }
    }
    child {
      node (E) [arn_x]{\small\!\! $merge(\frac{n}{4})$}
        child {
          node (EA) [arn_x]{$\ldots$}
          child {
                  node [arn_x]{$\frac{n}{2^h}$}
                }
          }
        child {
          node (EB) [arn_x]{$\ldots$}
          child {
                  node [arn_x]{$\frac{n}{2^h}$}
                }
          }
    }
  }
;
%\draw (BA) -- (BB);
%\draw (CA) -- (CB);
%\draw[<->] ($(BA)+(0,-0.3)$) -- ($(BB)+(0,-0.3)$) node[midway,below]{$2^h$};
%\draw[<->] ($(BA)+(-0.3,0)$) -- ($((BA).x-0.3,(B).y)$);
%\draw[<->]($(EB)+(0.3,0)$)--($(EB)+(0.3,3)$) node[midway,right]{$h+1$};
\draw[<->]($(BA)+(-0.3,-1.8)$)--($(BA)+(-0.3,4.5)$) node[midway,left]{$h$};
\path (4,1) node (t) {Total};
\path (4,0) node (n) {$n$};
\path (4,-1.5) node (n/2) {$n$};
\path (4,-3) node (n/4) {$n$};
\path (4,-6) node (base) {$n$};




\end{tikzpicture}

We check
\end{document}